% Options for packages loaded elsewhere
\PassOptionsToPackage{unicode}{hyperref}
\PassOptionsToPackage{hyphens}{url}
%
\documentclass[
]{article}
\usepackage{amsmath,amssymb}
\usepackage{lmodern}
\usepackage{iftex}
\ifPDFTeX
  \usepackage[T1]{fontenc}
  \usepackage[utf8]{inputenc}
  \usepackage{textcomp} % provide euro and other symbols
\else % if luatex or xetex
  \usepackage{unicode-math}
  \defaultfontfeatures{Scale=MatchLowercase}
  \defaultfontfeatures[\rmfamily]{Ligatures=TeX,Scale=1}
\fi
% Use upquote if available, for straight quotes in verbatim environments
\IfFileExists{upquote.sty}{\usepackage{upquote}}{}
\IfFileExists{microtype.sty}{% use microtype if available
  \usepackage[]{microtype}
  \UseMicrotypeSet[protrusion]{basicmath} % disable protrusion for tt fonts
}{}
\makeatletter
\@ifundefined{KOMAClassName}{% if non-KOMA class
  \IfFileExists{parskip.sty}{%
    \usepackage{parskip}
  }{% else
    \setlength{\parindent}{0pt}
    \setlength{\parskip}{6pt plus 2pt minus 1pt}}
}{% if KOMA class
  \KOMAoptions{parskip=half}}
\makeatother
\usepackage{xcolor}
\IfFileExists{xurl.sty}{\usepackage{xurl}}{} % add URL line breaks if available
\IfFileExists{bookmark.sty}{\usepackage{bookmark}}{\usepackage{hyperref}}
\hypersetup{
  pdftitle={R Markdown},
  pdfauthor={Mairead Bermingham},
  hidelinks,
  pdfcreator={LaTeX via pandoc}}
\urlstyle{same} % disable monospaced font for URLs
\usepackage[margin=1in]{geometry}
\usepackage{color}
\usepackage{fancyvrb}
\newcommand{\VerbBar}{|}
\newcommand{\VERB}{\Verb[commandchars=\\\{\}]}
\DefineVerbatimEnvironment{Highlighting}{Verbatim}{commandchars=\\\{\}}
% Add ',fontsize=\small' for more characters per line
\usepackage{framed}
\definecolor{shadecolor}{RGB}{248,248,248}
\newenvironment{Shaded}{\begin{snugshade}}{\end{snugshade}}
\newcommand{\AlertTok}[1]{\textcolor[rgb]{0.94,0.16,0.16}{#1}}
\newcommand{\AnnotationTok}[1]{\textcolor[rgb]{0.56,0.35,0.01}{\textbf{\textit{#1}}}}
\newcommand{\AttributeTok}[1]{\textcolor[rgb]{0.77,0.63,0.00}{#1}}
\newcommand{\BaseNTok}[1]{\textcolor[rgb]{0.00,0.00,0.81}{#1}}
\newcommand{\BuiltInTok}[1]{#1}
\newcommand{\CharTok}[1]{\textcolor[rgb]{0.31,0.60,0.02}{#1}}
\newcommand{\CommentTok}[1]{\textcolor[rgb]{0.56,0.35,0.01}{\textit{#1}}}
\newcommand{\CommentVarTok}[1]{\textcolor[rgb]{0.56,0.35,0.01}{\textbf{\textit{#1}}}}
\newcommand{\ConstantTok}[1]{\textcolor[rgb]{0.00,0.00,0.00}{#1}}
\newcommand{\ControlFlowTok}[1]{\textcolor[rgb]{0.13,0.29,0.53}{\textbf{#1}}}
\newcommand{\DataTypeTok}[1]{\textcolor[rgb]{0.13,0.29,0.53}{#1}}
\newcommand{\DecValTok}[1]{\textcolor[rgb]{0.00,0.00,0.81}{#1}}
\newcommand{\DocumentationTok}[1]{\textcolor[rgb]{0.56,0.35,0.01}{\textbf{\textit{#1}}}}
\newcommand{\ErrorTok}[1]{\textcolor[rgb]{0.64,0.00,0.00}{\textbf{#1}}}
\newcommand{\ExtensionTok}[1]{#1}
\newcommand{\FloatTok}[1]{\textcolor[rgb]{0.00,0.00,0.81}{#1}}
\newcommand{\FunctionTok}[1]{\textcolor[rgb]{0.00,0.00,0.00}{#1}}
\newcommand{\ImportTok}[1]{#1}
\newcommand{\InformationTok}[1]{\textcolor[rgb]{0.56,0.35,0.01}{\textbf{\textit{#1}}}}
\newcommand{\KeywordTok}[1]{\textcolor[rgb]{0.13,0.29,0.53}{\textbf{#1}}}
\newcommand{\NormalTok}[1]{#1}
\newcommand{\OperatorTok}[1]{\textcolor[rgb]{0.81,0.36,0.00}{\textbf{#1}}}
\newcommand{\OtherTok}[1]{\textcolor[rgb]{0.56,0.35,0.01}{#1}}
\newcommand{\PreprocessorTok}[1]{\textcolor[rgb]{0.56,0.35,0.01}{\textit{#1}}}
\newcommand{\RegionMarkerTok}[1]{#1}
\newcommand{\SpecialCharTok}[1]{\textcolor[rgb]{0.00,0.00,0.00}{#1}}
\newcommand{\SpecialStringTok}[1]{\textcolor[rgb]{0.31,0.60,0.02}{#1}}
\newcommand{\StringTok}[1]{\textcolor[rgb]{0.31,0.60,0.02}{#1}}
\newcommand{\VariableTok}[1]{\textcolor[rgb]{0.00,0.00,0.00}{#1}}
\newcommand{\VerbatimStringTok}[1]{\textcolor[rgb]{0.31,0.60,0.02}{#1}}
\newcommand{\WarningTok}[1]{\textcolor[rgb]{0.56,0.35,0.01}{\textbf{\textit{#1}}}}
\usepackage{graphicx}
\makeatletter
\def\maxwidth{\ifdim\Gin@nat@width>\linewidth\linewidth\else\Gin@nat@width\fi}
\def\maxheight{\ifdim\Gin@nat@height>\textheight\textheight\else\Gin@nat@height\fi}
\makeatother
% Scale images if necessary, so that they will not overflow the page
% margins by default, and it is still possible to overwrite the defaults
% using explicit options in \includegraphics[width, height, ...]{}
\setkeys{Gin}{width=\maxwidth,height=\maxheight,keepaspectratio}
% Set default figure placement to htbp
\makeatletter
\def\fps@figure{htbp}
\makeatother
\setlength{\emergencystretch}{3em} % prevent overfull lines
\providecommand{\tightlist}{%
  \setlength{\itemsep}{0pt}\setlength{\parskip}{0pt}}
\setcounter{secnumdepth}{-\maxdimen} % remove section numbering
\ifLuaTeX
  \usepackage{selnolig}  % disable illegal ligatures
\fi

\title{R Markdown}
\author{Mairead Bermingham}
\date{14/06/2022}

\begin{document}
\maketitle

\hypertarget{r-markdown}{%
\section{\texorpdfstring{\textbf{R
Markdown}}{R Markdown}}\label{r-markdown}}

R Markdown is a combination of Markdown (a lightweight markup language
for formatting text documents) and R code within it (usually to process
data, run analyses and produce tables and plots). R scripts have the
file extension .R, Markdown documents have a file extension .md;
therefore, R Markdown documents are .Rmd. Within RStudio there is a
Quick reference guide (figure 1) and links to the
\href{https://eu01.alma.exlibrisgroup.com/leganto/public/44UOE_INST/citation/37711420600002466?auth=SAML}{RStudio
cheatsheets} can be found in the Help drop-down menu.

\begin{figure}
\centering
\includegraphics[width=1\textwidth,height=\textheight]{../Figures/Images/RStudioMarkdownQuickReferenceGuide.png}
\caption{Figure 1. RStudio Markdown quick reference guide.}
\end{figure}

Here are three more helpful resources:

•
\href{https://eu01.alma.exlibrisgroup.com/leganto/public/44UOE_INST/citation/37711403400002466?auth=SAML}{The
R Markdown cheatsheet}

•
\href{https://eu01.alma.exlibrisgroup.com/leganto/public/44UOE_INST/citation/37711415410002466?auth=SAML}{The
R Markdown reference guide}

•
\href{https://eu01.alma.exlibrisgroup.com/leganto/public/44UOE_INST/citation/37711529450002466?auth=SAML}{R
Markdown quick tour}

\hypertarget{how-does-r-markdown-work}{%
\section{\texorpdfstring{\textbf{How does R Markdown
work?}}{How does R Markdown work?}}\label{how-does-r-markdown-work}}

Creating documents with R Markdown starts with an .Rmd document that
contains a combination of markdown (content with simple text formatting)
and R code chunks. The .Rmd document is fed to knitr, which executes all
of the R code chunks and creates a new markdown (.md) document which
includes the R code and its output (figure 2). The \emph{knitr} package
is an engine for dynamic report generation with R.

\begin{figure}
\centering
\includegraphics[width=1\textwidth,height=\textheight]{../Figures/Images/rmarkdownflow.png}
\caption{Figure 2. Markdown flow (source:
\href{https://eu01.alma.exlibrisgroup.com/leganto/public/44UOE_INST/citation/37711529450002466?auth=SAML}{RStudio,
2016}).}
\end{figure}

The markdown document generated by \emph{knitr} is then processed by
pandoc, which is responsible for creating the finished format. Pandoc is
a universal document converter.

This may sound not very easy, but R Markdown makes it extremely simple
by capturing all of the above processing into a single render function.
Better still, RStudio includes a ``Knit'' button that enables you to
render an .Rmd and preview it using a single click or keyboard shortcut.

\hypertarget{create-an-r-markdown-document}{%
\subsection{\texorpdfstring{\textbf{Create an R Markdown
document}}{Create an R Markdown document}}\label{create-an-r-markdown-document}}

To create an R Markdown document, go to the `File tab', then `New File',
and finally, select `R Markdown document' (figure 3).

\begin{figure}
\centering
\includegraphics[width=1\textwidth,height=\textheight]{../Figures/Images/CreateRMarkdownDocument.png}
\caption{Figure 3. Create an R Markdown document.}
\end{figure}

This is an R Markdown document, a plain text file with the extension
.Rmd (figure 4):

\begin{figure}
\centering
\includegraphics[width=1\textwidth,height=\textheight]{../Figures/Images/RMarkdownDocument.png}
\caption{Figure 4. R Markdown document.}
\end{figure}

The R Markdown document contains three types of content:

\begin{itemize}
\item
  An (optional) YAML header surrounded by ---s
\item
  R code chunks surrounded by ```s
\item
  text mixed with simple text formatting
\end{itemize}

\hypertarget{r-markdown-a-notebook-interface-for-r}{%
\subsection{\texorpdfstring{\textbf{R Markdown a notebook interface for
R}}{R Markdown a notebook interface for R}}\label{r-markdown-a-notebook-interface-for-r}}

When you open the R Markdown document in the RStudio, it becomes a
notebook interface for R. You can run each code chunk by clicking the
run icon (figure 5). RStudio, similar to the Jupyter Notebook, executes
the code and displays the results inline with your document:

\includegraphics[width=1\textwidth,height=\textheight]{../Figures/Images/RunCurrentChunk.png}

Lets run the chunk of code from my `MyFirstRMarkdownDocument' R Markdown
document below by clicking the run icon.

\begin{Shaded}
\begin{Highlighting}[]
\FunctionTok{summary}\NormalTok{(cars)}
\end{Highlighting}
\end{Shaded}

\begin{verbatim}
##      speed           dist       
##  Min.   : 4.0   Min.   :  2.00  
##  1st Qu.:12.0   1st Qu.: 26.00  
##  Median :15.0   Median : 36.00  
##  Mean   :15.4   Mean   : 42.98  
##  3rd Qu.:19.0   3rd Qu.: 56.00  
##  Max.   :25.0   Max.   :120.00
\end{verbatim}

\hypertarget{rendering-output}{%
\subsection{\texorpdfstring{\textbf{Rendering
output}}{Rendering output}}\label{rendering-output}}

To transform your R Markdown document into an HTML, PDF, or Word
document, click the ``Knit'' icon above your document in the scripts
editor (figure 6). A drop-down menu will let you select the type of
output you want. Lets select HTML. HTML (HyperText markup language) is
the code used to structure a web page and its content.

\begin{figure}
\centering
\includegraphics[width=1\textwidth,height=\textheight]{../Figures/Images/Knit.png}
\caption{Figure 6. Knit R Markdown document.png.}
\end{figure}

Here is how the MyFirstRMarkdownDocument R Markdown document would look
in HTML format:

\begin{figure}
\centering
\includegraphics[width=1\textwidth,height=\textheight]{../Figures/Images/MyFirstRMarkdownDocumentHTLMOutput.png}
\caption{Figure 7. The MyFirstRMarkdownDocument R Markdown document in
HTML format.}
\end{figure}

R Markdown generates a new document containing selected text, code, and
results from the .Rmd document. The output documents can be nicely
formatted PDFs, Word files, slideshows, and more (figure 8).

\begin{figure}
\centering
\includegraphics[width=0.5\textwidth,height=\textheight]{../Figures/Images/rmarkdownoutputformats.png}
\caption{Figure 8. Markdown output formats (source:
\href{https://eu01.alma.exlibrisgroup.com/leganto/public/44UOE_INST/citation/37711529450002466?auth=SAML}{RStudio,
2016})}
\end{figure}

R Markdown documents are fully reproducible and are designed to be used
in three ways:

\begin{enumerate}
\def\labelenumi{\arabic{enumi}.}
\item
  For communicating to decision-makers who want to focus on the
  conclusions, not the code behind the analysis.
\item
  For your future self and collaborators who are interested in your data
  insights and how you obtained them (i.e.~your data, procedures,
  results and code).
\item
  As a lab notebook, you can capture what you do and what you are
  thinking.
\end{enumerate}

\end{document}
